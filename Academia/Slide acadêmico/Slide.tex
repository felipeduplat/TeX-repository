
%%%%%%%%%%%%%%%%%%%%%%%%%%%%%%%%%%%%%%%%%%%%%%%%%%%%%%%%%%%%%%%%%%%%%%%%%%%%%%%

% UNIVERSIDADE FEDERAL DO PARANÁ (UFPR)
% SETOR DE CIÊNCIAS SOCIAIS APLICADAS
% PÓS-GRADUAÇÃO EM DESENVOLVIMENTO ECONÔMICO (PPGDE)
% DISCENTE: FELIPE DUPLAT LUZ

%%%%%%%%%%%%%%%%%%%%%%%%%%%%%%%%%%%%%%%%%%%%%%%%%%%%%%%%%%%%%%%%%%%%%%%%%%%%%%%

%%%%%% APRESENTAÇÃO DE SLIDES - PPGDE UFPR %%%%%%

% Classe:
\documentclass[aspectratio=169, 9pt]{beamer}


% Pacotes utilizados:
\usetheme{Darmstadt}                   % tema do slide.                
\usecolortheme{beaver}                 % cor do tema do slide.
\usepackage[style = abnt,              % manter formato ABNT.
            giveninits,                % manter primeiros nomes abreviados.
            scbib                      % manter em versalete.
           ]{biblatex}                 % adicionar referências.
\usepackage{tikz}                      % fazer o cabeçalho.
\usepackage[brazil]{babel}             % manter em PT-BR.
\usepackage{fontawesome}               % habilitar ícones sociais.
\usepackage{academicons}               % habilitar ícones acadêmicos.


% Configurações do beamer:
\setbeamertemplate{section in toc}[sections numbered]  % criar lista de conteúdo.
\setbeamertemplate{caption}[numbered]                  % numerar as legendas.
\setbeamerfont{caption}{size=\small}                   % tamanho da fonte das legendas.
\setbeamerfont{footnote}{size=\small}                  % tamanho da fonte do rodapé.
\setbeamerfont{footnotemark}{size=\small}              % tamanho da fonte das notas de rodapé.
\setbeamercovered{transparent}                         % tornar bullet points transparentes.
\setbeamertemplate{footline}{
  \leavevmode
  \hbox{\begin{beamercolorbox}[wd=.5\paperwidth,ht=4.5ex,dp=2.125ex,leftskip=.3cm plus1fill,rightskip=.3cm]{author in head/foot}
    \usebeamerfont{author in head/foot}\insertshortauthor
  \end{beamercolorbox}
  \begin{beamercolorbox}[wd=.5\paperwidth,ht=4.5ex,dp=2.125ex,leftskip=.3cm,rightskip=.3cm plus1fil]{title in head/foot}
    \usebeamerfont{title in head/foot}
    \parbox{.45\paperwidth}{\insertshorttitle\hfill \insertframenumber\,/\,\inserttotalframenumber}
  \end{beamercolorbox}}
  \vskip0pt%
}


% Referências:
\addbibresource{Referências.bib}


% Informações da capa:
\title[short-title]{Título da apresentação}
\author[Felipe Duplat Luz]{DECO0000 - Matéria \\ Prof. XXXX}
%\subtitle{}
\institute[UFPR] % (optional)
{
    \vspace{0.5mm} {\normalsize Felipe Duplat Luz} \vspace{5mm}
}
\date{XX de XXXX de \the\year{}}


% Cabeçalho:
\begin{document}

\begin{frame}[plain]
    \begin{center}
            \begin{minipage}[c]{0.2\linewidth}
                    \begin{center}
                    \includegraphics[width=2cm, height=1.7cm]{./Imagens/logo_UFPR.png} 
                    \end{center}
            \end{minipage}
            \begin{minipage}[c]{0.7\linewidth}
                    \begin{flushleft}
                    \begin{large}
                    Universidade Federal do Paraná - Departamento de Economia \\ \vspace{1mm} Setor de Ciências Sociais Aplicadas \\
                    \vspace{1mm} Pós-Graduação em Desenvolvimento Econômico
                    \end{large} 
                    \end{flushleft}
            \end{minipage}
    \end{center}
\titlepage 
\end{frame}


% Slide de conteúdos:
\begingroup
\setbeamertemplate{headline}{}
\addtobeamertemplate{frametitle}{\vspace*{-\headheight}}{}
\begin{frame}{Sumário}
	\tableofcontents[hideallsubsections]
\end{frame}
\endgroup


% Transição entre seções:
\AtBeginSection[]
{
	\begin{frame}{Sumário}
		\tableofcontents[currentsection, subsectionstyle=show/show/hide]
	\end{frame}
}


% Inserir todas as referências:
\nocite{*}



%%%%%%%%%%%%%%%%%%%%%%%%%%%%%%%%%%%%%%%%%%%%%%%%%%%%%%%%%%%%%%%%%%%%%%%%%%%%%%%


% DOCUMENTO COMEÇA A PARTIR DAQUI:


%%%%%%%%%%%%%%%%%%%%%%%%%%%%%%%%%%%%%%%%%%%%%%%%%%%%%%%%%%%%%%%%%%%%%%%%%%%%%%%


% Primeira seção:
\section{Seção 01}

\subsection[]{Subseção 01}

\begin{frame}{Subseção 01}
	\begin{itemize}[<+->]
		\item texto
		\item 
	\end{itemize}
\end{frame}


\subsection[]{Subseção 02}

\begin{frame}{Subseção 02}
	\begin{itemize}[<+->]
		\item texto
		\item 
	\end{itemize}
\end{frame}



% Segunda seção:
\section{Seção 02}

\subsection[]{Subseção 02}

\begin{frame}{Subseção 02}
	\begin{itemize}[<+->]
		\item texto
		\item 
	\end{itemize}
\end{frame}



% Terceira seção:
\section{Seção 03}

\subsection[]{Subseção 03}

\begin{frame}{Subseção 03}
	\begin{itemize}[<+->]
		\item texto
		\item 
	\end{itemize}
\end{frame}



% Referências:
\begin{frame}{Referências}
\printbibliography
\end{frame}


\begin{frame}
	\begin{center}
			\textbf{\textit{Obrigado}!} \\
			\vspace{0.5cm}
			
			\begin{tabular}{l}
				\faAt \hspace{0.001mm} \href{mailto:duplat.f@gmail.com}{duplat.f@gmail.com} \\
				\faLinkedin \hspace{0.001mm} \href{https://www.linkedin.com/in/felipeduplat/}{felipeduplat}
			\end{tabular}
		\end{center}
\end{frame}

\end{document}


