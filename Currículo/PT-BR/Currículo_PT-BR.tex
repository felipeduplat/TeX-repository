
%%%%%%%%%%%%%%%%%%%%%%%%%%%%%%%%%%%%%%%%%%%%%%%%%%%%%%%%%%%%%%%%%%%%%%%%%%%%%%%

%%% MODELO DE CURRÍCULO - FELIPE DUPLAT LUZ %%%

%%%%%%%%%%%%%%%%%%%%%%%%%%%%%%%%%%%%%%%%%%%%%%%%%%%%%%%%%%%%%%%%%%%%%%%%%%%%%%%

%% Classe:
\documentclass[10pt,a4paper]{cv}

%% Para símbolos:
% texdoc.net/pkg/fontawecome
% http://texdoc.net/pkg/academicons

%% Se compilar com XeLaTeX, deve-se inserir o Academicons.ttf na pasta de fontes do sistema

%% Configurações da página:
\geometry{left=1cm,right=9cm,marginparwidth=6.8cm,marginparsep=0.8cm,top=1cm,bottom=1cm}
\renewcommand{\itemmarker}{{\small\textbullet}}
\renewcommand{\ratingmarker}{\faCircle}

%% Pacotes utilizados:
\usepackage[utf8]{inputenc}  % fontes
\usepackage[T1]{fontenc}     % fontes
\usepackage[default]{lato}   % fontes
\usepackage{academicons}     % habilitar ícones acadêmicos
\usepackage{hyperref}        % habilitar hyperlink

%% Alterar as cores:
\definecolor{VividPurple}{HTML}{2E64FE}
\definecolor{SlateGrey}{HTML}{2E2E2E}
\definecolor{LightGrey}{HTML}{666666}
\colorlet{heading}{VividPurple}
\colorlet{accent}{VividPurple}
\colorlet{emphasis}{SlateGrey}
\colorlet{body}{LightGrey}

%% Inserir publicações:
\addbibresource{publicacoes.bib}



%%%%%%%%%%%%%%%%%%%%%%%%%%%%%%%%%%%%%%%%%%%%%%%%%%%%%%%%%%%%%%%%%%%%%%%%%%%%%%%


% DOCUMENTO COMEÇA A PARTIR DAQUI:


%%%%%%%%%%%%%%%%%%%%%%%%%%%%%%%%%%%%%%%%%%%%%%%%%%%%%%%%%%%%%%%%%%%%%%%%%%%%%%%

\begin{document}
\name{Felipe Duplat Luz}
\tagline{}
\photo{2.5cm}{Imagens/001.jpg}
\personalinfo{
  % pode inserir os próprios com \printinfo{symbol}{detail}
  \phone{+55 (XX) XXXXX-XXXX}
  \location{Cidade/UF} \\
  \email{\href{mailto:duplat.f@gmail.com}{duplat.f@gmail.com}}
  \printinfo{\aiLattes}{\href{http://lattes.cnpq.br/7966980343097478}{Lattes}}
  \orcid{\href{https://orcid.org/0000-0002-5808-1611}{Orcid}}
  \github{\href{https://github.com/felipeduplat}{GitHub}}
  \linkedin{\href{https://www.linkedin.com/in/felipeduplat/}{LinkedIn}}
}

%% Cabeçalho se estender durante toda a extensão da direita - 8cm (=6.8cm marginparwidth + 1.2cm marginparsep)
\begin{adjustwidth}{}{-7.6cm}
\makecvheader
\end{adjustwidth}



%% EXPERIÊNCIA
\cvsection[Coluna]{Experiência}

\cvevent{\small Bolsista de Incentivo à Pesquisa}{Instituto de Pesquisa Econômica Aplicada (Ipea)}{maio 23 -- atual}{Brasília/DF}
\begin{itemize}
	\item Desenvolvimento de algoritmos e \textit{scripts} computacionais para indicadores sobre trabalho, previdência, saúde e educação.
\end{itemize}

\divider

\cvevent{\small Estagiário}{Federação das Indústrias do Estado da Bahia (FIEB)}{out 20 -- mar 22}{Salvador/BA}
\begin{itemize}
\item Elaboração e apresentação de relatórios técnicos sobre a construção civil no estado da Bahia.
\end{itemize}

\divider

\cvevent{\small Estagiário}{Companhia de Desenvolvimento e Ação Regional (CAR)}{mar 19 -- ago 19}{Salvador/BA}
\begin{itemize}
\item Análise e acompanhamento de projetos de inclusão socioprodutiva.
\end{itemize}

\divider

\cvevent{\small Estagiário}{Superintendência De Estudos Econômicos E Sociais da Bahia (SEI)}{jul 16-- jan 17}{Salvador/BA}
\begin{itemize}
	\item Elaboração de relatórios mensais sobre desemprego local e especiais sobre grupos focais (jovens, idosos, mulheres e negros).
\end{itemize}



%% EDUCAÇÃO
\cvsection{Educação}

\cvevent{\small Mestrado \ em Desenvolvimento Econômico}{Universidade Federal do Paraná (UFPR)}{mar 22 -- atual}{Curitiba/PR}

\divider

\cvevent{\small Graduação \ em Ciências Econômicas}{Universidade Federal da Bahia (UFBA)}{jun 16 -- dez 21}{Salvador/BA}

\divider

\cvevent{\small Intercâmbio Acadêmico em Economia}{Universidade de Santiago de Compostela (USC)}{ago 19 -- jun 20}{Santiago de Compostela/ESP}



%% PUBLICAÇÔES
\cvsection{Atividades acadêmicas}

\textbf{Apresentação de trabalho} -- Análise da desigualdade de renda entre brancos e negros no estado da Bahia, VIII Semana de Estudos Socioeconômicos, 2018 - UEFS

\textbf{Curso de extensão} -- Introdução ao \LaTeX{}, 2023 - Universidade Federal do Paraná (UFPR)

\textbf{Capítulos de livros publicados} -- DUPLAT, F; MENDES, V. Efeito da pandemia de COVID-19 sobre a pobreza e desigualdade de renda no Brasil no ano de 2020. In: PASSOS, H. D.; PESSOTI, G. C. (Org.). Reflexões de Economistas Baianos 2023. 19ª ed. Salvador: CORECON-BA, 2023.



%% ATIVIDADES EXTRACURRICULARES
%\clearpage
%\cvsection{Atividades extracurriculares}

%\cvachievement{\faUniversity}{Associação Atlética Acadêmica de Economia da UFBA}{Vice-presidente (2019-2020)}

%\divider

%\cvachievement{\faUniversity}{Diretório Acadêmico Plínio Moura}{Diretor Político (2016) e Diretor Geral (2017-2019)}

%\divider

%\cvachievement{\faBook}{Biblioteca Central do Estado da Bahia}{Leitor para cegos (2014-2016)}

\end{document}


