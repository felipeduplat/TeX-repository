
%%%%%%%%%%%%%%%%%%%%%%%%%%%%%%%%%%%%%%%%%%%%%%%%%%%%%%%%%%%%%%%%%%%%%%%%%%%%%%%

% UNIVERSIDADE FEDERAL DO PARANÁ (UFPR)
% PÓS-GRADUAÇÃO EM DESENVOLVIMENTO ECONÔMICO (PPGDE)
% FACULDADE DE ECONOMIA
% DISCENTE: FELIPE DUPLAT LUZ

%%%%%%%%%%%%%%%%%%%%%%%%%%%%%%%%%%%%%%%%%%%%%%%%%%%%%%%%%%%%%%%%%%%%%%%%%%%%%%%

%%%%%% TRABALHO ACADÊMICO - PPGDE UFPR %%%%%%

% Classe:
\documentclass[a4paper, 11pt]{article}


% Pacotes utilizados:
\usepackage[brazil]{babel}             % manter em PT-BR.
\usepackage{tikz}                      % fazer o cabeçalho.
\usepackage{fancyhdr}                  % fazer o rodapé.
\usepackage{titlesec}                  % automatizar as seções.
\usepackage[style = abnt-numeric,      % manter formato ABNT.
			giveninits,                % manter primeiros nomes abreviados.
			scbib                      % manter em versalete.
		   ]{biblatex}                 % adicionar referências.


% Configurações do documento:
\usepackage[bindingoffset=0.2in,
			left=1in,
			right=1in,
			top=1in,
			bottom=1in,
			footskip=.25in]{geometry}
\setlength{\textwidth}{18cm} 
\setlength{\oddsidemargin}{-1cm}


% Referências:
\addbibresource{Referências.bib}


% Cabeçalho:
\setlength\parindent{0pt}
\begin{document}
	
	% linha superior:
	\begin{tikzpicture}
		\draw[thick] (-6.5,0)--(10,0);
	\end{tikzpicture}
	
	\colorbox{white!10!}{
		\begin{minipage}[l]{0.169 \textwidth}
			\begin{flushleft}
				\includegraphics[width=1.1in,height=0.85in]{./Imagens/logo_UFPR.png}
			\end{flushleft}
		\end{minipage}
		\begin{minipage}[l]{0.75 \textwidth}
			\begin{flushleft}
				{\large \textsc{Universidade Federal do Paraná - Faculdade de Economia}}
				\\
				{\large \textsc{Pós Graduação em Desenvolvimento Econômico}}
				\\
				\large \textsc{DECO0000 - []}
				\\
				\large \textsc{Docente: []}
				\\
				\large \textsc{Discente: Felipe Duplat Luz}
			\end{flushleft}
		\end{minipage}
	}
	
	% linha inferior:
	\begin{tikzpicture}
		\draw[thick] (-6.5,0)--(10,0);
	\end{tikzpicture}


% Rodapé:
\fancyhf{}
\pagestyle{fancy}
\renewcommand{\footrulewidth}{0.1mm}
\renewcommand{\headrulewidth}{0.0mm}
\fancyfoot[R]{\thepage}

\fancypagestyle{plain}{
	\fancyhf{}
	\renewcommand{\footrulewidth}{0.1mm}
	\fancyfoot[R]{\thepage}
	\renewcommand{\headrulewidth}{empty}}


% Automatização das seções:
\titleformat{\section}{\large\bfseries}{\underline{Questão \thesection}}{0.1cm}{}[]


%% Inserir todas as referências:
\nocite{sotomayor21, dolevchenko09}



%%%%%%%%%%%%%%%%%%%%%%%%%%%%%%%%%%%%%%%%%%%%%%%%%%%%%%%%%%%%%%%%%%%%%%%%%%%%%%%


% TEXTO COMEÇA A PARTIR DAQUI:


%%%%%%%%%%%%%%%%%%%%%%%%%%%%%%%%%%%%%%%%%%%%%%%%%%%%%%%%%%%%%%%%%%%%%%%%%%%%%%%


% Título do trabalho:
	\begin{center}
		\huge [título do trabalho]
	\end{center}

% Questão 01:
\section{}

[texto]


% Questão 02:
\section{}

[texto]


% Questão 03:
\section{}

[texto]



% Referências:
\pagebreak
\printbibliography[title={\underline{Referências:}}]

\end{document}


