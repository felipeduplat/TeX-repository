
% packages, font, color, and new commands
\usepackage{amsfonts, amsmath, oldgerm, lmodern, animate}
\usepackage{verbatim}
\usepackage{bm}
\usepackage{multirow}     % quebrar célula na tabela.
\usepackage[style = abnt, % manter formato ABNT.
giveninits,               % manter primeiros nomes abreviados.
scbib                     % manter em versalete.
]{biblatex}               % adicionar referências.
\usefonttheme{serif}
\usepackage[flushleft]{threeparttable} % inserir notas nas tabelas

% TYPESETTING ELEMENTS

% style of section presented in the table of contents
\setbeamertemplate{section in toc}{$\blacktriangleright$~\inserttocsection}

% style of subsection presented in the table of contents
\setbeamertemplate{subsection in toc}{}
\setbeamertemplate{subsection in toc}{\footnotesize\hspace{1.2 em}$\blacktriangleright$~\inserttocsubsection}
\setbeamercovered{transparent}

% automate subtitle of each frame
\makeatletter
    \pretocmd\beamer@checkframetitle{\framesubtitle{\thesection \, \secname}}
\makeatother

% avoid numbering of frames that are breaked into multiply slides
\setbeamertemplate{frametitle continuation}{}

% at the begining of section, add table of contents with the current section highlighted
\AtBeginSection[]
{
    \begingroup
    \setbeamertemplate{footline}{}
    \themecolor{blue}
    \begin{frame}{Sumário}
        \tableofcontents[currentsection, subsectionstyle=show/show/hide]
    \end{frame}
    \endgroup
}

% at the beginning of subsection, add subsection title page
\AtBeginSubsection[]
{
    \begin{frame}{\,}{\thesection \, \secname}
        \fontfamily{ptm}\selectfont
        \centering\textsl{\textbf{\textcolor{sintefblue}{
            \vskip15pt
            \Large \subsecname
        }}}
    \end{frame}
}

% code bolck setting
\definecolor{codegreen}{RGB}{101,218,120}
\definecolor{codegray}{rgb}{0.5,0.5,0.5}
\definecolor{codepurple}{rgb}{0.58,0,0.82}
\definecolor{backcolour}{rgb}{0.95,0.95,0.92}

\lstdefinestyle{mystyle}{
    % backgroundcolor=\color{backcolour},
    commentstyle=\color{airforceblue},
    keywordstyle=\color{magenta},
    numberstyle=\tiny\color{codegray},
    stringstyle=\color{codepurple},
    basicstyle=\ttfamily\scriptsize,
    breakatwhitespace=false,
    breaklines=true,
    captionpos=b,
    keepspaces=true,
    numbers=left,
    numbersep=5pt,
    showspaces=false,
    showstringspaces=false,
    showtabs=false,
    tabsize=4,
    xleftmargin=10pt,
    xrightmargin=10pt,
}

\lstset{style=mystyle}

% NEW COMMANDS

% set colored hyperlinks command
\newcommand{\hrefcol}[2]{\textcolor{airforceblue}{\href{#1}{#2}}}
\newcommand{\hlinkcol}[1]{\hrefcol{#1}{#1}}


% centering paragraph statement
\newcommand{\centerstate}[1]{
    \centering
    \begin{columns}
        \begin{column}{0.8\textwidth}
            #1
        \end{column}
    \end{columns}
}

% colored textbf
\newcommand{\ctextbf}[1]{\textbf{\textcolor{sintefblue}{#1}}}
\newcommand{\btextbf}[1]{\textbf{\textcolor{airforceblue}{#1}}}

% colored textsl
\newcommand{\ctextsl}[1]{\textsl{\textcolor{sintefblue}{#1}}}
\newcommand{\btextsl}[1]{\textsl{\textcolor{airforceblue}{#1}}}

% colored emph
\newcommand{\cemph}[1]{\emph{\textcolor{sintefblue}{#1}}}
\newcommand{\bemph}[1]{\emph{\textcolor{airforceblue}{#1}}}

% about page
\newcommand{\aboutpage}[2]{
    \begingroup
    \setbeamertemplate{footline}{}
    \themecolor{blue}
    \begin{frame}[c]{#1}{\,}
        \centering
        \begin{minipage}{\textwidth}
            \usebeamercolor[fg]{normal text}
            \centering
            \Large \textsl{\normalsize #2}
        \end{minipage}
    \end{frame}
    \endgroup
}

% bibliography page
\newcommand{\bibliographpage}{
    \section{Referências}
    \begingroup
    \themecolor{blue}
    \begin{frame}[allowframebreaks]{Referências}{\,}
        \tiny
        \printbibliography[heading=none]
    \end{frame}
\endgroup
}