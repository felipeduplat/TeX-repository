
% ----------------------------------------------------------
% PACOTES BÁSICOS
% ----------------------------------------------------------

\usepackage[T1]{fontenc}		     % seleção de códigos de fonte.
\usepackage[utf8]{inputenc}		     % codificação do documento.
\usepackage{lastpage}		     	 % para a Ficha catalográfica.
\usepackage{indentfirst}     		 % indenta o primeiro parágrafo de cada seção.
\usepackage{color}		         	 % controle das cores.
\usepackage{graphicx}	     		 % inclusão de gráficos.
\usepackage{microtype} 	     		 % para melhorias de justificação.
\usepackage{ifthen}		         	 % para montar condicionais.
\usepackage[brazil]{babel}	     	 % para utilizar termos em portugues.
\usepackage[final]{pdfpages}         % para incluir páginas de arquivos pdf.
\usepackage{lipsum}				     % para gerar dummy text.
\usepackage{csquotes}                % para citações.
%\usepackage[style=long]{glossaries} % para glossário.
%\usepackage{abntex2glossaries}      % para glossário.
\usepackage{cancel} 		         % cancelamento de termos em texto ou equações.
\usepackage{xcolor} 		         % cores estendidas.
\usepackage{smartdiagram}   	     % gera diagramas a partir de listas.
\usepackage{float} 		             % para a figura ficar na posição correta.	    
\usepackage{textcomp} 		         % suporte para fontes da Text Companion. 
\usepackage{longtable}		         % uso de longtable.
\usepackage{amsmath}		         % símbolos matemáticos.
\usepackage{pdflscape}		         % páginas em paisagem.
\usepackage{multicol}		         % mescla de colunas em tabelas.
\usepackage{multirow}		         % mescla de linhas em tabelas.
\usepackage{newfloat} 		         % criação do indice de quadros.
\usepackage{slashbox}                % para tabelas.
\usepackage{makecell}                % para tabelas.
\usepackage{threeparttable}          % para tabelas.
\usepackage{nameref}                 % para labels dos capítulos.
\usepackage{changepage}              % alterar margem das tabelas.
\usepackage{caption}                 % alterar título das tabelas.
\usepackage{longtable}               % permitir que a tabela atravesse várias páginas.
\usepackage{makecell}                % quebrar linha dentro das células de uma tabela.
\usepackage[cal=esstix]{mathalfa}    % fontes matemáticas.
\usepackage{bbm}                     % símbolos matemáticos.
\usepackage{dcolumn}                 % para tabelas das regressões.
\usepackage{fancyhdr}				 % tirar cabeçalho.
\usepackage{titlesec}                % ajustar seções.

% para equações matemáticas:
\DeclareMathOperator\Log{Log}

% Para legendas:
%\usepackage{caption}
%[format=plain]
%\renewcommand\caption[1]{
%\captionsetup{font=small}
%,format=hang
%\caption{#1}}
%\captionsetup{width=0.8\textwidth}
\captiondelim{-- }
\captiontitlefont{\small}
\captionnamefont{\small}

% não resetar notas de rodapé:
\counterwithout{footnote}{chapter}



% ----------------------------------------------------------
% FORMATAÇÃO
% ----------------------------------------------------------

% BibLaTeX:
\usepackage[
style=abnt,
backref=true,
backend=biber,
citecounter=true,
backrefstyle=three, 
url=true,
maxbibnames=99,
mincitenames=1,
maxcitenames=3,
backref=false,
hyperref=true,
giveninits=true,
uniquename=false,
uniquelist=false]{biblatex}

% Espaçamento entre os itens nas referências:
%\setlength\bibitemsep{\baselineskip}

% Texto padrão para as referências:
\DefineBibliographyStrings{brazil}{%
	 backrefpage  = {Citado \arabic{citecounter} vez na página},
	 backrefpages = {Citado \arabic{citecounter} vezes nas páginas},
	 urlfrom      = {Dispon\'ivel em},
}

% Indentação de referências:
\defbibheading{bay}[\bibname]{
\chapter*{#1}
\markboth{#1}{#1}%
\addcontentsline{toc}{chapter}
%{\protect\numberline{}\bibname}
{\bibname}}

% Formatando o avanço dos títulos no sumário:
\makeatletter
	\pretocmd{\chapter}{\addtocontents{toc}{\protect\addvspace{-12\p@}}}{}{}
	\pretocmd{\section}{\addtocontents{toc}{\protect\addvspace{-3\p@}}}{}{}
	\pretocmd{\chapter}{\addtocontents{lof}{\protect\vspace{-10\p@}}}{}{}
	\pretocmd{\chapter}{\addtocontents{loq}{\protect\vspace{-10\p@}}}{}{}
	\pretocmd{\chapter}{\addtocontents{lot}{\protect\vspace{-8\p@}}}{}{}
\makeatother


% Inserir maiúsculo na seção:
% (https://groups.google.com/g/abntex2/c/ZYwE4t9uTFM)
\makeatletter
\let\oldcontentsline\contentsline
\def\contentsline#1#2{%
	\expandafter\ifx\csname l@#1\endcsname\l@section
	\expandafter\@firstoftwo
	\else
	\expandafter\@secondoftwo
	\fi
	{%
		\oldcontentsline{#1}{\MakeTextUppercase{#2}}%
	}{%
		\oldcontentsline{#1}{{#2}}%
	}%
}
\makeatother

% Retirar os símbolos <> da URL:
\DeclareFieldFormat{illustrated}{\addspace #1\isdot}
%\DeclareFieldFormat{url}{\bibstring{urlform}\addcolon\addspace<\url{#1}>}
%\DeclareFieldFormat{url}{\bibstring{urlfrom}\addcolon\addspace<\url{#1}>}
\DeclareFieldFormat{url}{\bibstring{urlfrom}\addcolon \space\addspace{#1}}

% Ajustar o espaço para a formatação da data:
\DeclareFieldFormat{urldate}{\bibstring{urlseen}\addcolon\addspace #1}%
\DeclareFieldFormat*{note}{\addspace #1}%

% Ajustar o tamanho da fonte do número da primeira página do capítulo (na parte textual):
\makepagestyle{chapfirst}
\makeoddhead{chapfirst}{}{}{\footnotesize{\thepage}}

% Novo estilo de cabeçalhos e rodapés:
\makepagestyle{simplestextual}

% Cabeçalho para página par:
\makeevenhead{simplestextual}
  {}{}{\footnotesize \thepage}

% Cabeçalho para página ímpar:
\makeoddhead{simplestextual} %%pagina ímpar ou com oneside
  {}{}{\footnotesize \thepage}

% Linha no rodapé:  
%\makeheadrule{simplestextual}{\textwidth}{\normalrulethickness}

% Rodapé:
\makeevenfoot{simplestextual}
  {}{}{} %%pagina par

% Página ímpar ou com oneside:      
\makeoddfoot{simplestextual}
  {}{}{}

% Formatação dos capítulos pós-textuais numerados:
%\newcommand{\refap}[1]{\hyperref[#1]{Apêndice~\ref{#1}}} 	% Referência apÊndices

% uso do tikz e pgfplots:
%\usetikzlibrary{external}
\usetikzlibrary{arrows,calc,patterns,angles,quotes}
\usepackage{pgfplots}
\pgfplotsset{compat=1.15}

% Define o comando para citação de fontes em elementos gráficos:
\newcommand{\citefig}[2]{~\Citeauthor*{#1}\citeyear{#1}}

% Define os operadores matemáticos em português:
\DeclareMathOperator{\tr}{tr}
\DeclareMathOperator{\sen}{sen}
\DeclareMathOperator{\senh}{senh}
%\DeclareMathOperator{\tag}{tag}
\DeclareMathOperator{\tg}{tg}
\DeclareMathOperator{\tagh}{tagh}
\DeclareMathOperator{\tgh}{tgh}
\DeclareMathOperator{\cossec}{cossec}
%\DeclareMathOperator{\sen}{sen}

% Listagem de codigos LaTeX na documentação:
\usepackage{listings}

% Citação de documentos não publicados e informais e colocar nas notas de rodapé:
\newcommand{\citenp}[1]{
\cite{#1}\footnote{\fullcite{#1}}}

\newcommand{\textcitenp}[1]{
	\textcite{#1}\footnote{\fullcite{#1}}}

% Tirar negrito do \citetitle:
\DeclareFieldFormat{citetitle}{#1}


