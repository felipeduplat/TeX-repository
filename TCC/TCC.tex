
%%%%%%%%%%%%%%%%%%%%%%%%%%%%%%%%%%%%%%%%%%%%%%%%%

% UNIVERSIDADE FEDERAL DO PARANÁ (UFPR)
% SETOR DE CIÊNCIAS SOCIAIS APLICADAS
% PÓS-GRADUAÇÃO EM DESENVOLVIMENTO ECONÔMICO (PPGDE)
% DISCENTE: FELIPE DUPLAT LUZ

%%%%%%%%%%%%%%%%%%%%%%%%%%%%%%%%%%%%%%%%%%%%%%%%%

%%%%%% TRABALHO DE CONCLUSÃO DE CURSO (MONOGRAFIA, DISSERTAÇÃO OU TESE) %%%%%%

%----------------------------------------------------------------
%% Classe abntex2.cls:
%% abntex2.cls, v-1.9.5 laurocesar
%% Copyright 2012-2015 by abnTeX2 group at https://www.abntex.net.br/ 
%%
%----------------------------------------------------------------

% Classe:
\documentclass[
12pt,            % tamanho da fonte.
openright,	     % capítulos começam em pág ímpar (insere página vazia caso preciso).
oneside,         % para impressão em páginas separadas (somente anverso).
%twoside,        % para impressão em anverso (frente) e verso.
a4paper,         % tipo do papel.
chapter=TITLE,   % títulos de capítulos convertidos em letras maiúsculas.
section=TITLE,   % títulos de seções convertidos em letras maiúsculas.
english,         % ENG para hifenização.
spanish,         % ESP para hifenização.
brazil,          % PT-BR para hifenização - o último idioma é o principal do documento
dvipsnames       % cores adicionais.
]{abntex2}

% Estilo do documento:
\usepackage{Arquivos/UFPR}

% Pacotes e formatação:

% ----------------------------------------------------------
% PACOTES BÁSICOS
% ----------------------------------------------------------

\usepackage[T1]{fontenc}		     % seleção de códigos de fonte.
\usepackage[utf8]{inputenc}		     % codificação do documento.
\usepackage{lastpage}		     	 % para a Ficha catalográfica.
\usepackage{indentfirst}     		 % indenta o primeiro parágrafo de cada seção.
\usepackage{color}		         	 % controle das cores.
\usepackage{graphicx}	     		 % inclusão de gráficos.
\usepackage{microtype} 	     		 % para melhorias de justificação.
\usepackage{ifthen}		         	 % para montar condicionais.
\usepackage[brazil]{babel}	     	 % para utilizar termos em portugues.
\usepackage[final]{pdfpages}         % para incluir páginas de arquivos pdf.
\usepackage{lipsum}				     % para gerar dummy text.
\usepackage{csquotes}                % para citações.
%\usepackage[style=long]{glossaries} % para glossário.
%\usepackage{abntex2glossaries}      % para glossário.
\usepackage{cancel} 		         % cancelamento de termos em texto ou equações.
\usepackage{xcolor} 		         % cores estendidas.
\usepackage{smartdiagram}   	     % gera diagramas a partir de listas.
\usepackage{float} 		             % para a figura ficar na posição correta.	    
\usepackage{textcomp} 		         % suporte para fontes da Text Companion. 
\usepackage{longtable}		         % uso de longtable.
\usepackage{amsmath}		         % símbolos matemáticos.
\usepackage{pdflscape}		         % páginas em paisagem.
\usepackage{multicol}		         % mescla de colunas em tabelas.
\usepackage{multirow}		         % mescla de linhas em tabelas.
\usepackage{newfloat} 		         % criação do indice de quadros.
\usepackage{slashbox}                % para tabelas.
\usepackage{makecell}                % para tabelas.
\usepackage{threeparttable}          % para tabelas.
\usepackage{nameref}                 % para labels dos capítulos.
\usepackage{changepage}              % alterar margem das tabelas.
\usepackage{caption}                 % alterar título das tabelas.
\usepackage{longtable}               % permitir que a tabela atravesse várias páginas.
\usepackage{makecell}                % quebrar linha dentro das células de uma tabela.
\usepackage[cal=esstix]{mathalfa}    % fontes matemáticas.
\usepackage{bbm}                     % símbolos matemáticos.
\usepackage{dcolumn}                 % para tabelas das regressões.
\usepackage{fancyhdr}				 % tirar cabeçalho.
\usepackage{titlesec}                % ajustar seções.

% para equações matemáticas:
\DeclareMathOperator\Log{Log}

% Para legendas:
%\usepackage{caption}
%[format=plain]
%\renewcommand\caption[1]{
%\captionsetup{font=small}
%,format=hang
%\caption{#1}}
%\captionsetup{width=0.8\textwidth}
\captiondelim{-- }
\captiontitlefont{\small}
\captionnamefont{\small}

% não resetar notas de rodapé:
\counterwithout{footnote}{chapter}



% ----------------------------------------------------------
% FORMATAÇÃO
% ----------------------------------------------------------

% BibLaTeX:
\usepackage[
style=abnt,
backref=true,
backend=biber,
citecounter=true,
backrefstyle=three, 
url=true,
maxbibnames=99,
mincitenames=1,
maxcitenames=3,
backref=false,
hyperref=true,
giveninits=true,
uniquename=false,
uniquelist=false]{biblatex}

% Espaçamento entre os itens nas referências:
%\setlength\bibitemsep{\baselineskip}

% Texto padrão para as referências:
\DefineBibliographyStrings{brazil}{%
	 backrefpage  = {Citado \arabic{citecounter} vez na página},
	 backrefpages = {Citado \arabic{citecounter} vezes nas páginas},
	 urlfrom      = {Dispon\'ivel em},
}

% Indentação de referências:
\defbibheading{bay}[\bibname]{
\chapter*{#1}
\markboth{#1}{#1}%
\addcontentsline{toc}{chapter}
%{\protect\numberline{}\bibname}
{\bibname}}

% Formatando o avanço dos títulos no sumário:
\makeatletter
	\pretocmd{\chapter}{\addtocontents{toc}{\protect\addvspace{-12\p@}}}{}{}
	\pretocmd{\section}{\addtocontents{toc}{\protect\addvspace{-3\p@}}}{}{}
	\pretocmd{\chapter}{\addtocontents{lof}{\protect\vspace{-10\p@}}}{}{}
	\pretocmd{\chapter}{\addtocontents{loq}{\protect\vspace{-10\p@}}}{}{}
	\pretocmd{\chapter}{\addtocontents{lot}{\protect\vspace{-8\p@}}}{}{}
\makeatother


% Inserir maiúsculo na seção:
% (https://groups.google.com/g/abntex2/c/ZYwE4t9uTFM)
\makeatletter
\let\oldcontentsline\contentsline
\def\contentsline#1#2{%
	\expandafter\ifx\csname l@#1\endcsname\l@section
	\expandafter\@firstoftwo
	\else
	\expandafter\@secondoftwo
	\fi
	{%
		\oldcontentsline{#1}{\MakeTextUppercase{#2}}%
	}{%
		\oldcontentsline{#1}{{#2}}%
	}%
}
\makeatother

% Retirar os símbolos <> da URL:
\DeclareFieldFormat{illustrated}{\addspace #1\isdot}
%\DeclareFieldFormat{url}{\bibstring{urlform}\addcolon\addspace<\url{#1}>}
%\DeclareFieldFormat{url}{\bibstring{urlfrom}\addcolon\addspace<\url{#1}>}
\DeclareFieldFormat{url}{\bibstring{urlfrom}\addcolon \space\addspace{#1}}

% Ajustar o espaço para a formatação da data:
\DeclareFieldFormat{urldate}{\bibstring{urlseen}\addcolon\addspace #1}%
\DeclareFieldFormat*{note}{\addspace #1}%

% Ajustar o tamanho da fonte do número da primeira página do capítulo (na parte textual):
\makepagestyle{chapfirst}
\makeoddhead{chapfirst}{}{}{\footnotesize{\thepage}}

% Novo estilo de cabeçalhos e rodapés:
\makepagestyle{simplestextual}

% Cabeçalho para página par:
\makeevenhead{simplestextual}
  {}{}{\footnotesize \thepage}

% Cabeçalho para página ímpar:
\makeoddhead{simplestextual} %%pagina ímpar ou com oneside
  {}{}{\footnotesize \thepage}

% Linha no rodapé:  
%\makeheadrule{simplestextual}{\textwidth}{\normalrulethickness}

% Rodapé:
\makeevenfoot{simplestextual}
  {}{}{} %%pagina par

% Página ímpar ou com oneside:      
\makeoddfoot{simplestextual}
  {}{}{}

% Formatação dos capítulos pós-textuais numerados:
%\newcommand{\refap}[1]{\hyperref[#1]{Apêndice~\ref{#1}}} 	% Referência apÊndices

% uso do tikz e pgfplots:
%\usetikzlibrary{external}
\usetikzlibrary{arrows,calc,patterns,angles,quotes}
\usepackage{pgfplots}
\pgfplotsset{compat=1.15}

% Define o comando para citação de fontes em elementos gráficos:
\newcommand{\citefig}[2]{~\Citeauthor*{#1}\citeyear{#1}}

% Define os operadores matemáticos em português:
\DeclareMathOperator{\tr}{tr}
\DeclareMathOperator{\sen}{sen}
\DeclareMathOperator{\senh}{senh}
%\DeclareMathOperator{\tag}{tag}
\DeclareMathOperator{\tg}{tg}
\DeclareMathOperator{\tagh}{tagh}
\DeclareMathOperator{\tgh}{tgh}
\DeclareMathOperator{\cossec}{cossec}
%\DeclareMathOperator{\sen}{sen}

% Listagem de codigos LaTeX na documentação:
\usepackage{listings}

% Citação de documentos não publicados e informais e colocar nas notas de rodapé:
\newcommand{\citenp}[1]{
\cite{#1}\footnote{\fullcite{#1}}}

\newcommand{\textcitenp}[1]{
	\textcite{#1}\footnote{\fullcite{#1}}}

% Tirar negrito do \citetitle:
\DeclareFieldFormat{citetitle}{#1}




% Informações do documento:

% ----------------------------------------------------------
% DADOS
% ----------------------------------------------------------

% Tipo de TCC (Monografia, Dissertação, Tese ou Relatório Técnico):
\tipotrabalho{Dissertação}

% Informações do TCC:
\titulo{título do trabalho}              % título.
\autor{Felipe Duplat Luz}                % autor.
\local{Curitiba}                         % cidade.
\data{2023}                              % ano.
\orientador{Vinícius de Almeida Vale}    % orientador  - se homem.
%\orientadora{}                          % orientadora - se mulher.
%\coorientador{}                         % co-orientador  - se homem
\coorientadora{Kênia Barreiro de Souza}  % co-orientadora - se mulher
%\scoorientador{}                        % segundo co-orientador  - se mulher
%\scoorientadora{}                       % segunda co-orientadora - se mulher

% Adicionar referências:
\addbibresource{Arquivos/Referências.bib}

% Cabeçalho da capa:
\instituicao{Universidade Federal do Paraná}
\def \ImprimirSetor{Setor de Ciências Sociais Aplicadas}
\def \ImprimirProgramaPos{Programa de Pós-Graduação em Desenvolvimento Econômico}
\def \ImprimirCurso{}
\preambulo{Projeto de dissertação de mestrado apresentada ao Programa de Pós-Graduação em Desenvolvimento Econômico do Setor de Ciências Sociais Aplicadas da Universidade Federal do Paraná como requisito para obtenção do título de mestre em Desenvolvimento Econômico}

% Informações complementares:
\newcommand{\imprimirCurso}{}
\newcommand{\imprimirDataDefesa}{06 de março de 2024}
\newcommand{\imprimircdu}{02:141:005.7}

% Comandos de dados - Data da apresentação
\providecommand{\imprimirdataapresentacaoRotulo}{}
\providecommand{\imprimirdataapresentacao}{}
\newcommand{\dataapresentacao}[2][\dataapresentacaoname]{\renewcommand{\dataapresentacao}{#2}}

% Comandos de dados - Nome do Curso
\providecommand{\imprimirnomedocursoRotulo}{}
\providecommand{\imprimirnomedocurso}{}
\newcommand{\nomedocurso}[2][\nomedocursoname]
  {\renewcommand{\imprimirnomedocursoRotulo}{#1}
\renewcommand{\imprimirnomedocurso}{#2}}






% ----------------------------------------------------------
% DADOS INICIAIS
% ----------------------------------------------------------

\begin{document}

% Uppercase dos títulos:
\renewcommand{\tablename}{TABELA}
\renewcommand{\figurename}{FIGURA}
\renewcommand{\figureautorefname}{FIGURA}
\renewcommand{\tableautorefname}{TABELA}
\newcommand{\equationname}{EQUA\c{C}\~AO~}
\renewcommand{\equationautorefname}{EQUA\c{C}\~AO~}
\renewcommand{\bibname}{{REFER\^ENCIAS}}
\renewcommand{\apendicesname}{Ap\^endices}
\renewcommand{\anexosname}{Anexos}

% Fonte do número da primeira página do capítulo:
\aliaspagestyle{chapter}{chapfirst}



% ----------------------------------------------------------
% ELEMENTOS PRÉ-TEXTUAIS
% ----------------------------------------------------------
\pretextual


%---------------------------------------------------------------------
% CAPA
%---------------------------------------------------------------------

\renewcommand{\imprimircapa}{%
	\begin{capa}%
		\center
		\begin{tikzpicture}[remember picture,overlay] 
			\node[anchor=south west, yshift= 25mm, xshift=-1.5mm] at 
			(current page.south west) 
			{\includegraphics[width = \paperwidth]{Imagens/capa_UFPR}};
		\end{tikzpicture}
		\center
		%    \ABNTEXchapterfont 
		\MakeUppercase\imprimirinstituicao \vspace{-2mm}
		
		%\ifthenelse{\equal \ImprimirSetor{}}{}{
			%    \ABNTEXchapterfont 
		%	\MakeUppercase\ImprimirSetor}
		%\vspace{-2mm}
		
		%\ifthenelse{\equal \ImprimirProgramaPos{}}{}{
		%	     \ABNTEXchapterfont 
		%	\MakeUppercase\ImprimirProgramaPos}
		
		\ifthenelse{\equal \ImprimirCurso{}}{}{
			%    \ABNTEXchapterfont 
			\MakeUppercase\ImprimirCurso}
		
		\vspace{40mm}
		
		%    \ABNTEXchapterfont 
		\MakeUppercase\imprimirautor
		
		\vspace{40mm}
		%    \ABNTEXchapterfont
		\MakeUppercase\imprimirtitulo
		\vfill
		
		%\large
		\MakeUppercase\imprimirlocal
		
		%\large
		\imprimirdata
		
		\vspace*{10mm}
	\end{capa}
}

\pdfbookmark[0]{Capa}{Capa}
\imprimircapa


                             % Capa.

%---------------------------------------------------------------------
% FOLHA DE ROSTO
%---------------------------------------------------------------------

\makeatletter
\renewcommand{\folhaderostocontent}{
	\begin{center}
		
		%\vspace*{1cm}
		{
			%\ABNTEXchapterfont
			%\large
			\MakeUppercase\imprimirautor}
		
		\vspace*{\fill}%\vspace*{\fill}
		\begin{center}
			%      \ABNTEXchapterfont
			%\bfseries
			%\Large
			\MakeUppercase\imprimirtitulo
		\end{center}
		\vspace*{\fill}
		
		\abntex@ifnotempty{\imprimirpreambulo}{%
			\hspace{.45\textwidth}
			\begin{minipage}{.5\textwidth}
				\SingleSpacing\small
				\imprimirpreambulo.\vspace*{2mm}
				
				\abntex@ifnotempty{\imprimirorientador}
				{\imprimirorientadorRotulo~\imprimirorientador}
				\ifthenelse{\equal{\imprimirorientador}{}}
				{\imprimirorientadoraRotulo~\imprimirorientadora}
				{}
				\abntex@ifnotempty{\imprimircoorientador}
				{\par\imprimircoorientadorRotulo~\imprimircoorientador}%
				{\abntex@ifnotempty{\imprimircoorientadora}
					{\par\imprimircoorientadoraRotulo~\imprimircoorientadora}%
				}
				\ifthenelse{\equal{\imprimirscoorientadora}{} \AND \equal{\imprimirscoorientador}{}}{}
				{
					\ifthenelse{\equal{\imprimirscoorientador}{}}{}
					{\par\imprimirscoorientadorRotulo~\imprimirscoorientador}
					\ifthenelse{\equal{\imprimirscoorientadora}{}}{}
					{\par\imprimirscoorientadoraRotulo~\imprimirscoorientadora}
				}%
				%
			\end{minipage}%
			\vspace*{\fill}
		}%
		
		\vspace*{\fill}
		
		
		{  \MakeUppercase\imprimirlocal}
		\par
		{    \imprimirdata}
		\vspace*{1cm}
		
	\end{center}
}
\makeatother

\imprimirfolhaderosto*


                   % Folha de rosto.

%---------------------------------------------------------------------
% Ficha catalográfica
%---------------------------------------------------------------------

\begin{fichacatalografica}
	\hspace{-1.4cm}
	\imprimirnotaautorizacao \\ \\
	%\sffamily
	\vspace*{\fill}				% posição vertical
\begin{center}					% minipage Centralizado
  \imprimirnotabib
  \begin{table}[htb]
	\scriptsize
	\centering	
	\begin{tabular}{|p{0.9cm} p{8.7cm}|}
		\hline
	      & \\
		  &	  \imprimirautorficha     \\
		
		 \imprimircutter & 
							\hspace{0.4cm}\imprimirtitulo~  / ~\imprimirautor~ ;  ~\imprimirorientadorcorpoficha. -- 	\imprimirlocal, \imprimirdata.   \\
		
		  &   % Para incluir nota referente à versão corrigida no corpo da ficha,
			  % incluir % no início da linha acima e tirar a % do início da linha abaixo
			  %	\hspace{0.4cm} \imprimirtitulo~  / ~\imprimirautor~ ; ~\imprimirorientadorcorpoficha~- ~\imprimirnotafolharosto. -- \imprimirlocal, \imprimirdata.  \\
		
			\hspace{0.4cm}\pageref{LastPage} p. : il. (algumas color.) ; 30 cm.\\ 
		  & \\
		  & 
		    \hspace{0.4cm}\imprimirnotaficha ~--~ 
						  \imprimirunidademin, 
						  \imprimiruniversidademin, 
		                  \imprimirdata. \\ 
		  & \\                 
		    % Para incluir nota referente à versão corrigida em notas,
		    % incluir uma % no início da linha acima e	
		    % tirar a % do início da linha abaixo
		    % & \hspace{0.4cm}\imprimirnotafolharosto \\ 
		  & \\ 
		  & \hspace{0.4cm}1. LaTeX. 2. abnTeX. 3. Classe USPSC. 4. Editoração de texto. 5. Normalização da documentação. 6. Tese. 7. Dissertação. 8. Documentos (elaboração). 9. Documentos eletrônicos. I. \imprimirorientadorficha. 
		   %II. Título. \\ % se não tiver co-orientador
		   III. Título. \\ % se tiver co-orientador
		  \hline
	\end{tabular}
  \end{table}
\end{center}
\end{fichacatalografica}


              % Ficha catalográfica.
%\include{Pré/Errata}                          % Errata.

%---------------------------------------------------------------------
% TERMO DE APROVAÇÃO
%---------------------------------------------------------------------

% São duas opções possíveis: usar um PDF pronto ou usar o comando abaixo. Se quiser usar o PDF, basta substituir o arquivo .pdf na pasta "Pré". Caso queira usar o comando, basta editá-lo de acordo com suas necessidades e depois remover o .pdf da pasta "Pré".

\newcommand{\insereAprovacao}{
	\IfFileExists{Pré/Termo_de_aprovação.pdf}
	{\includepdf[pages=-]{Pré/Termo_de_aprovação.pdf}}
	{
		\begin{folhadeaprovacao}%\color{blue}
			
			\begin{center}
				{\ABNTEXchapterfont
					{\large\bfseries\MakeUppercase\folhadeaprovacaoname}\par\phantom{}\par
					%\large
					\MakeUppercase\imprimirautor}
				
				\vspace*{\fill}\vspace*{\fill}
				\begin{center}
					\ABNTEXchapterfont
					%\bfseries\Large
					\MakeUppercase\imprimirtitulo
				\end{center}
				\vspace*{\fill}
				\begin{minipage}{\textwidth}
					\hspace{.45\textwidth}
					\begin{minipage}{.5\textwidth}
						\imprimirpreambulo pela seguinte banca examinadora:
					\end{minipage}%
				\end{minipage}
				
				
				\vspace*{\fill}
			\end{center}
			\assinatura{{
					\ifthenelse{\equal{\imprimirorientador}{}}
					{\imprimirorientadora \\ Orientadora}
					{\imprimirorientador \\ Orientador}}
			}
			%\assinatura{
			%	{Kênia Barreiro de Souza} \\ Co-orientadora}
			\assinatura{%\textbf
				{Pessoa 01} \\ Avaliador interno (Universidade X)}
			\assinatura{%\textbf
				{Pessoa 02} \\ Avaliador externo (Universidade Y)}
			%\assinatura{%\textbf{Professor} \\ Convidado 4}
			
			\begin{center}
				\vspace*{0.5cm}
				%{\large\imprimirlocal}
				%\par
				%{\large\imprimirdata}
				\imprimirlocal, \imprimirDataDefesa.
				\vspace*{1cm}
			\end{center}
			
		\end{folhadeaprovacao}
	}
}

\insereAprovacao


               % Termo de aprovação.
%
%---------------------------------------------------------------------
% DEDICATÓRIA
%---------------------------------------------------------------------

\begin{dedicatoria}
	\vspace*{\fill}
	\centering
	\noindent
	
	\textit{Eu dedico...}
	
	\vspace*{\fill}
\end{dedicatoria}


                     % Dedicatória.

%---------------------------------------------------------------------
% AGRADECIMENTOS
%---------------------------------------------------------------------

\begin{agradecimentos}
	Eu agradeço...
\end{agradecimentos}


                   % Agradecimentos.
%
%---------------------------------------------------------------------
% EPÍGRAFE
%---------------------------------------------------------------------

\begin{epigrafe}
	\vspace*{\fill}
	\begin{flushright}
		\textit{``as  elites são patrimonialistas e conservadoras, a classe média \\ é meritocrática e o povo? o povo não é nada!''}
	\end{flushright}
\end{epigrafe}


                        % Epígrafe.

%---------------------------------------------------------------------
% RESUMO
%---------------------------------------------------------------------

% PT-BR:
\begin{resumo}
	\SingleSpacing
	
	Texto.
	
	\noindent 
	\textbf{Palavras-chave}: . \\
	\textbf{Classificação JEL}: .
\end{resumo}


                           % Resumo.

%---------------------------------------------------------------------
% ABSTRACT
%---------------------------------------------------------------------

% ENG:
\begin{resumo}[Abstract]
	\begin{otherlanguage*}{english}
		\SingleSpacing
		
		Abstract of the text.
		
		\noindent 
		\textbf{Keywords}: Word 01. Word 02. Word 03.
	\end{otherlanguage*}
\end{resumo}


                         % Abstract.

%---------------------------------------------------------------------
% LISTAS
%---------------------------------------------------------------------

% Lista de figuras:
\pdfbookmark[0]{\listfigurename}{lof}
\listoffigures*
\cleardoublepage



% Lista de quadros:
\pdfbookmark[0]{\listtablename}{lot}
\listofquadros*
\cleardoublepage



% Lista de tabelas:
\pdfbookmark[0]{\listtablename}{lot}
\listoftables*
\cleardoublepage



% Lista de abreviaturas e símbolos:
\begin{siglas}
	
	\item[ECV] Esporte Clube Vitória.
	
	\item[]     
	
\end{siglas}


% Símbolos:
\begin{simbolos}
	\item[$\alpha$] Letra grega Alfa em minúsculo.
	\item[$\beta$]  Letra grega Beta em minúsculo.
	\item[$\gamma$] Letra grega gama em minúsculo.
\end{simbolos}



% Sumário:
%\pdfbookmark[0]{\contentsname}{toc}
\phantompart \tableofcontents*
%\cleardoublepage


                           % Listas de figuras, quadros, tabelas e sumário.



% ----------------------------------------------------------
% ELEMENTOS TEXTUAIS
% ----------------------------------------------------------
\textual


% ----------------------------------------------------------
% CAPÍTULO 01 - INTRODUÇÃO
% ----------------------------------------------------------

\chapter{Introdução} \label{cha:introdução}

\section{Teste} \label{sec:teste}

Texto \cite{anderson20, campostimini22, do09}.




% ----------------------------------------------------------
% CAPÍTULO 02 - Revisão de literatura
% ----------------------------------------------------------

\chapter{Revisão de literatura}

Texto.

% ----------------------------------------------------------
% CAPÍTULO 03 - METODOLOGIA E DADOS
% ----------------------------------------------------------

\chapter{Metodologia e dados}

Texto.


	


% ----------------------------------------------------------
% CAPÍTULO 04 - RESULTADOS
% ----------------------------------------------------------

\chapter{Resultados}

Texto.


	


% ----------------------------------------------------------
% CAPÍTULO 05 - CONSIDERAÇÕES FINAIS
% ----------------------------------------------------------

\chapter{Considerações finais}

Texto.

	




% ----------------------------------------------------------
% ELEMENTOS PÓS-TEXTUAIS
% ----------------------------------------------------------
\postextual

% Referências:
\begingroup
\renewcommand*{\bibfont}{\raggedleft}
\printbibliography[heading = bibintoc, notkeyword = {consulta}, notkeyword={npub-informal}]
\endgroup

% Capítulos pós-textuais:
%\addtocontents{toc}{\vspace{4pt}}

% Apêndice:

%---------------------------------------------------------------------
% APÊNDICE
%---------------------------------------------------------------------

\begin{apendicesenv}
	%\renewcommand{\thechapter}{\arabic{chapter}}
	%\partapendices

	\chapter{Arquivos da dissertação} \label{ap:a'}

	Os arquivos desse \textit{template} podem ser acessados no repositório do \href{https://github.com/felipeduplat/TeX-repository}{GitHub}.

\end{apendicesenv}




% Anexo:
%
%---------------------------------------------------------------------
% ANEXO
%---------------------------------------------------------------------

\begin{anexosenv}
	\renewcommand{\thechapter}{\arabic{chapter}}
	\partanexos
	\chapter{Título} \label{an:an01}
	
	Texto.
	
\end{anexosenv}




\end{document}


